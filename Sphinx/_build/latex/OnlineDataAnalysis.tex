% Generated by Sphinx.
\def\sphinxdocclass{report}
\documentclass[letterpaper,10pt,english]{sphinxmanual}
\usepackage[utf8]{inputenc}
\DeclareUnicodeCharacter{00A0}{\nobreakspace}
\usepackage{cmap}
\usepackage[T1]{fontenc}
\usepackage{babel}
\usepackage{times}
\usepackage[Bjarne]{fncychap}
\usepackage{longtable}
\usepackage{sphinx}
\usepackage{multirow}

\addto\captionsenglish{\renewcommand{\figurename}{Fig. }}
\addto\captionsenglish{\renewcommand{\tablename}{Table }}
\floatname{literal-block}{Listing }



\title{Online Data Analysis Documentation}
\date{October 13, 2015}
\release{1.0.0}
\author{Alastair Chin}
\newcommand{\sphinxlogo}{}
\renewcommand{\releasename}{Release}
\makeindex

\makeatletter
\def\PYG@reset{\let\PYG@it=\relax \let\PYG@bf=\relax%
    \let\PYG@ul=\relax \let\PYG@tc=\relax%
    \let\PYG@bc=\relax \let\PYG@ff=\relax}
\def\PYG@tok#1{\csname PYG@tok@#1\endcsname}
\def\PYG@toks#1+{\ifx\relax#1\empty\else%
    \PYG@tok{#1}\expandafter\PYG@toks\fi}
\def\PYG@do#1{\PYG@bc{\PYG@tc{\PYG@ul{%
    \PYG@it{\PYG@bf{\PYG@ff{#1}}}}}}}
\def\PYG#1#2{\PYG@reset\PYG@toks#1+\relax+\PYG@do{#2}}

\expandafter\def\csname PYG@tok@mh\endcsname{\def\PYG@tc##1{\textcolor[rgb]{0.13,0.50,0.31}{##1}}}
\expandafter\def\csname PYG@tok@cm\endcsname{\let\PYG@it=\textit\def\PYG@tc##1{\textcolor[rgb]{0.25,0.50,0.56}{##1}}}
\expandafter\def\csname PYG@tok@sr\endcsname{\def\PYG@tc##1{\textcolor[rgb]{0.14,0.33,0.53}{##1}}}
\expandafter\def\csname PYG@tok@se\endcsname{\let\PYG@bf=\textbf\def\PYG@tc##1{\textcolor[rgb]{0.25,0.44,0.63}{##1}}}
\expandafter\def\csname PYG@tok@nn\endcsname{\let\PYG@bf=\textbf\def\PYG@tc##1{\textcolor[rgb]{0.05,0.52,0.71}{##1}}}
\expandafter\def\csname PYG@tok@gd\endcsname{\def\PYG@tc##1{\textcolor[rgb]{0.63,0.00,0.00}{##1}}}
\expandafter\def\csname PYG@tok@go\endcsname{\def\PYG@tc##1{\textcolor[rgb]{0.20,0.20,0.20}{##1}}}
\expandafter\def\csname PYG@tok@gi\endcsname{\def\PYG@tc##1{\textcolor[rgb]{0.00,0.63,0.00}{##1}}}
\expandafter\def\csname PYG@tok@gr\endcsname{\def\PYG@tc##1{\textcolor[rgb]{1.00,0.00,0.00}{##1}}}
\expandafter\def\csname PYG@tok@il\endcsname{\def\PYG@tc##1{\textcolor[rgb]{0.13,0.50,0.31}{##1}}}
\expandafter\def\csname PYG@tok@vc\endcsname{\def\PYG@tc##1{\textcolor[rgb]{0.73,0.38,0.84}{##1}}}
\expandafter\def\csname PYG@tok@gs\endcsname{\let\PYG@bf=\textbf}
\expandafter\def\csname PYG@tok@nl\endcsname{\let\PYG@bf=\textbf\def\PYG@tc##1{\textcolor[rgb]{0.00,0.13,0.44}{##1}}}
\expandafter\def\csname PYG@tok@ow\endcsname{\let\PYG@bf=\textbf\def\PYG@tc##1{\textcolor[rgb]{0.00,0.44,0.13}{##1}}}
\expandafter\def\csname PYG@tok@w\endcsname{\def\PYG@tc##1{\textcolor[rgb]{0.73,0.73,0.73}{##1}}}
\expandafter\def\csname PYG@tok@mi\endcsname{\def\PYG@tc##1{\textcolor[rgb]{0.13,0.50,0.31}{##1}}}
\expandafter\def\csname PYG@tok@gu\endcsname{\let\PYG@bf=\textbf\def\PYG@tc##1{\textcolor[rgb]{0.50,0.00,0.50}{##1}}}
\expandafter\def\csname PYG@tok@c\endcsname{\let\PYG@it=\textit\def\PYG@tc##1{\textcolor[rgb]{0.25,0.50,0.56}{##1}}}
\expandafter\def\csname PYG@tok@sb\endcsname{\def\PYG@tc##1{\textcolor[rgb]{0.25,0.44,0.63}{##1}}}
\expandafter\def\csname PYG@tok@o\endcsname{\def\PYG@tc##1{\textcolor[rgb]{0.40,0.40,0.40}{##1}}}
\expandafter\def\csname PYG@tok@gp\endcsname{\let\PYG@bf=\textbf\def\PYG@tc##1{\textcolor[rgb]{0.78,0.36,0.04}{##1}}}
\expandafter\def\csname PYG@tok@vg\endcsname{\def\PYG@tc##1{\textcolor[rgb]{0.73,0.38,0.84}{##1}}}
\expandafter\def\csname PYG@tok@ne\endcsname{\def\PYG@tc##1{\textcolor[rgb]{0.00,0.44,0.13}{##1}}}
\expandafter\def\csname PYG@tok@cp\endcsname{\def\PYG@tc##1{\textcolor[rgb]{0.00,0.44,0.13}{##1}}}
\expandafter\def\csname PYG@tok@s2\endcsname{\def\PYG@tc##1{\textcolor[rgb]{0.25,0.44,0.63}{##1}}}
\expandafter\def\csname PYG@tok@nt\endcsname{\let\PYG@bf=\textbf\def\PYG@tc##1{\textcolor[rgb]{0.02,0.16,0.45}{##1}}}
\expandafter\def\csname PYG@tok@kc\endcsname{\let\PYG@bf=\textbf\def\PYG@tc##1{\textcolor[rgb]{0.00,0.44,0.13}{##1}}}
\expandafter\def\csname PYG@tok@nc\endcsname{\let\PYG@bf=\textbf\def\PYG@tc##1{\textcolor[rgb]{0.05,0.52,0.71}{##1}}}
\expandafter\def\csname PYG@tok@ni\endcsname{\let\PYG@bf=\textbf\def\PYG@tc##1{\textcolor[rgb]{0.84,0.33,0.22}{##1}}}
\expandafter\def\csname PYG@tok@s1\endcsname{\def\PYG@tc##1{\textcolor[rgb]{0.25,0.44,0.63}{##1}}}
\expandafter\def\csname PYG@tok@nb\endcsname{\def\PYG@tc##1{\textcolor[rgb]{0.00,0.44,0.13}{##1}}}
\expandafter\def\csname PYG@tok@kr\endcsname{\let\PYG@bf=\textbf\def\PYG@tc##1{\textcolor[rgb]{0.00,0.44,0.13}{##1}}}
\expandafter\def\csname PYG@tok@cs\endcsname{\def\PYG@tc##1{\textcolor[rgb]{0.25,0.50,0.56}{##1}}\def\PYG@bc##1{\setlength{\fboxsep}{0pt}\colorbox[rgb]{1.00,0.94,0.94}{\strut ##1}}}
\expandafter\def\csname PYG@tok@sc\endcsname{\def\PYG@tc##1{\textcolor[rgb]{0.25,0.44,0.63}{##1}}}
\expandafter\def\csname PYG@tok@ss\endcsname{\def\PYG@tc##1{\textcolor[rgb]{0.32,0.47,0.09}{##1}}}
\expandafter\def\csname PYG@tok@kt\endcsname{\def\PYG@tc##1{\textcolor[rgb]{0.56,0.13,0.00}{##1}}}
\expandafter\def\csname PYG@tok@mb\endcsname{\def\PYG@tc##1{\textcolor[rgb]{0.13,0.50,0.31}{##1}}}
\expandafter\def\csname PYG@tok@m\endcsname{\def\PYG@tc##1{\textcolor[rgb]{0.13,0.50,0.31}{##1}}}
\expandafter\def\csname PYG@tok@sh\endcsname{\def\PYG@tc##1{\textcolor[rgb]{0.25,0.44,0.63}{##1}}}
\expandafter\def\csname PYG@tok@sx\endcsname{\def\PYG@tc##1{\textcolor[rgb]{0.78,0.36,0.04}{##1}}}
\expandafter\def\csname PYG@tok@sd\endcsname{\let\PYG@it=\textit\def\PYG@tc##1{\textcolor[rgb]{0.25,0.44,0.63}{##1}}}
\expandafter\def\csname PYG@tok@gh\endcsname{\let\PYG@bf=\textbf\def\PYG@tc##1{\textcolor[rgb]{0.00,0.00,0.50}{##1}}}
\expandafter\def\csname PYG@tok@mf\endcsname{\def\PYG@tc##1{\textcolor[rgb]{0.13,0.50,0.31}{##1}}}
\expandafter\def\csname PYG@tok@mo\endcsname{\def\PYG@tc##1{\textcolor[rgb]{0.13,0.50,0.31}{##1}}}
\expandafter\def\csname PYG@tok@k\endcsname{\let\PYG@bf=\textbf\def\PYG@tc##1{\textcolor[rgb]{0.00,0.44,0.13}{##1}}}
\expandafter\def\csname PYG@tok@c1\endcsname{\let\PYG@it=\textit\def\PYG@tc##1{\textcolor[rgb]{0.25,0.50,0.56}{##1}}}
\expandafter\def\csname PYG@tok@err\endcsname{\def\PYG@bc##1{\setlength{\fboxsep}{0pt}\fcolorbox[rgb]{1.00,0.00,0.00}{1,1,1}{\strut ##1}}}
\expandafter\def\csname PYG@tok@vi\endcsname{\def\PYG@tc##1{\textcolor[rgb]{0.73,0.38,0.84}{##1}}}
\expandafter\def\csname PYG@tok@no\endcsname{\def\PYG@tc##1{\textcolor[rgb]{0.38,0.68,0.84}{##1}}}
\expandafter\def\csname PYG@tok@na\endcsname{\def\PYG@tc##1{\textcolor[rgb]{0.25,0.44,0.63}{##1}}}
\expandafter\def\csname PYG@tok@ge\endcsname{\let\PYG@it=\textit}
\expandafter\def\csname PYG@tok@nf\endcsname{\def\PYG@tc##1{\textcolor[rgb]{0.02,0.16,0.49}{##1}}}
\expandafter\def\csname PYG@tok@kn\endcsname{\let\PYG@bf=\textbf\def\PYG@tc##1{\textcolor[rgb]{0.00,0.44,0.13}{##1}}}
\expandafter\def\csname PYG@tok@kd\endcsname{\let\PYG@bf=\textbf\def\PYG@tc##1{\textcolor[rgb]{0.00,0.44,0.13}{##1}}}
\expandafter\def\csname PYG@tok@bp\endcsname{\def\PYG@tc##1{\textcolor[rgb]{0.00,0.44,0.13}{##1}}}
\expandafter\def\csname PYG@tok@nd\endcsname{\let\PYG@bf=\textbf\def\PYG@tc##1{\textcolor[rgb]{0.33,0.33,0.33}{##1}}}
\expandafter\def\csname PYG@tok@nv\endcsname{\def\PYG@tc##1{\textcolor[rgb]{0.73,0.38,0.84}{##1}}}
\expandafter\def\csname PYG@tok@kp\endcsname{\def\PYG@tc##1{\textcolor[rgb]{0.00,0.44,0.13}{##1}}}
\expandafter\def\csname PYG@tok@s\endcsname{\def\PYG@tc##1{\textcolor[rgb]{0.25,0.44,0.63}{##1}}}
\expandafter\def\csname PYG@tok@si\endcsname{\let\PYG@it=\textit\def\PYG@tc##1{\textcolor[rgb]{0.44,0.63,0.82}{##1}}}
\expandafter\def\csname PYG@tok@gt\endcsname{\def\PYG@tc##1{\textcolor[rgb]{0.00,0.27,0.87}{##1}}}

\def\PYGZbs{\char`\\}
\def\PYGZus{\char`\_}
\def\PYGZob{\char`\{}
\def\PYGZcb{\char`\}}
\def\PYGZca{\char`\^}
\def\PYGZam{\char`\&}
\def\PYGZlt{\char`\<}
\def\PYGZgt{\char`\>}
\def\PYGZsh{\char`\#}
\def\PYGZpc{\char`\%}
\def\PYGZdl{\char`\$}
\def\PYGZhy{\char`\-}
\def\PYGZsq{\char`\'}
\def\PYGZdq{\char`\"}
\def\PYGZti{\char`\~}
% for compatibility with earlier versions
\def\PYGZat{@}
\def\PYGZlb{[}
\def\PYGZrb{]}
\makeatother

\renewcommand\PYGZsq{\textquotesingle}

\begin{document}

\maketitle
\tableofcontents
\phantomsection\label{index::doc}


\includegraphics{UWA-logo.png}

This contains the documentation for the online data analysis project


\chapter{Requirements}
\label{index:requirements}\label{index:welcome-to-online-data-analysis-s-documentation}
This requires Python 3, scipy, numpy and pandas which can be installed
by using the Anaconda package which contains all necessary files.

The program currently consists of 6 files

Contents:


\section{Application}
\label{Code_rst/app:module-application}\label{Code_rst/app::doc}\label{Code_rst/app:application}\index{application (module)}
Main execution body for program.
\index{get\_file\_dir() (in module application)}

\begin{fulllineitems}
\phantomsection\label{Code_rst/app:application.get_file_dir}\pysiglinewithargsret{\code{application.}\bfcode{get\_file\_dir}}{\emph{location}}{}
Returns the directory of the file with the file name
\begin{description}
\item[{Keyword arguments:}] \leavevmode
location -- A file path.

\end{description}

\end{fulllineitems}

\index{main() (in module application)}

\begin{fulllineitems}
\phantomsection\label{Code_rst/app:application.main}\pysiglinewithargsret{\code{application.}\bfcode{main}}{\emph{*args}}{}
Create Data and Report objects, providing necessary information for them 
to run analysis and create desired outputs (i.e. HTML report).

\end{fulllineitems}



\section{Data}
\label{Code_rst/dat:module-data}\label{Code_rst/dat::doc}\label{Code_rst/dat:data}\index{data (module)}
Reads CSV file for information, provides basic cleaning of data and then
runs analysis on said data.
\index{Column (class in data)}

\begin{fulllineitems}
\phantomsection\label{Code_rst/dat:data.Column}\pysiglinewithargsret{\strong{class }\code{data.}\bfcode{Column}}{\emph{header='`}}{}
Object to hold data from each column within the provided CSV file.
\begin{description}
\item[{Methods:}] \leavevmode
change\_misc\_values -- Removes misc/unclear values from column values.

drop\_greater\_than -- Removes `\textless{}', `\textgreater{}' from column values.

define\_most\_least\_common -- Sets object variable to hold 15 most common values
and least common values for that column.

define\_type -- Sets object variable to type (e.g., String) according
to column values.

define\_errors -- Defines a list that contains the row and column of possibly
incorrect values.

\item[{Variables:}] \leavevmode
most\_common -- \textless{}= 15 most common results within the column values.

least\_common -- \textless{}= 15 least common results within the column values.

empty -- Boolean value of whether the column holds values or not.

header -- Column header/title.

type -- The type of data in column, e.g., String, Float, Integer,
Enumerated.

values -- List of CSV values for the column.

analysis -- Analysis object associated with this column.

\end{description}
\index{change\_misc\_values() (data.Column method)}

\begin{fulllineitems}
\phantomsection\label{Code_rst/dat:data.Column.change_misc_values}\pysiglinewithargsret{\bfcode{change\_misc\_values}}{}{}
Replaces identified values of unclear meaning or inexact value, i.e., 
`-`, with an agreed value.

\end{fulllineitems}

\index{define\_errors() (data.Column method)}

\begin{fulllineitems}
\phantomsection\label{Code_rst/dat:data.Column.define_errors}\pysiglinewithargsret{\bfcode{define\_errors}}{\emph{columnNumber}, \emph{errors}, \emph{formatted\_errors}, \emph{invalid\_rows\_pos}, \emph{range\_list2}, \emph{set\_to\_ignore}}{}
Define all the rows/columns with invalid values and append to errors, and
formatted\_errors once formatted properly. invalid\_rows\_pos holds the amount of
rows that have been skipped by the time the current row x is being considered.
\begin{description}
\item[{Keyword arguments:}] \leavevmode
columnNumber -- The number of the current column being iterated over, numbered
from 0.

errors -- A list of errors to be edited of the form (row number, column number,
error value) which is numbered from 0.

formatted\_errors -- A list of errors to be edited of the form (row number, column
number, error value) which is numbered from 1.

invalid\_rows\_pos -- An array containing a number matching the amount of invalid
rows that have been removed from analysis by the time that row is accessed. i.e.
invalid\_rows\_pos{[}1{]} = 2 says that by the time values{[}1{]} is evaluated two rows have
been removed from analysis.

\end{description}

\end{fulllineitems}

\index{define\_most\_least\_common() (data.Column method)}

\begin{fulllineitems}
\phantomsection\label{Code_rst/dat:data.Column.define_most_least_common}\pysiglinewithargsret{\bfcode{define\_most\_least\_common}}{}{}
Set 15 most common results to class variable, and set object variable 
empty if appropriate.

\end{fulllineitems}

\index{define\_type() (data.Column method)}

\begin{fulllineitems}
\phantomsection\label{Code_rst/dat:data.Column.define_type}\pysiglinewithargsret{\bfcode{define\_type}}{}{}
Run column data against regex filters and assign object variable type
as appropriate.

\end{fulllineitems}

\index{is\_Empty() (data.Column method)}

\begin{fulllineitems}
\phantomsection\label{Code_rst/dat:data.Column.is_Empty}\pysiglinewithargsret{\bfcode{is\_Empty}}{}{}
Whether or not column is empty

\end{fulllineitems}

\index{set\_Identifier\_size() (data.Column method)}

\begin{fulllineitems}
\phantomsection\label{Code_rst/dat:data.Column.set_Identifier_size}\pysiglinewithargsret{\bfcode{set\_Identifier\_size}}{\emph{size}}{}
Sets the size of the data for identifier type

\end{fulllineitems}

\index{set\_empty() (data.Column method)}

\begin{fulllineitems}
\phantomsection\label{Code_rst/dat:data.Column.set_empty}\pysiglinewithargsret{\bfcode{set\_empty}}{}{}
Set Column to be empty

\end{fulllineitems}

\index{set\_not\_empty() (data.Column method)}

\begin{fulllineitems}
\phantomsection\label{Code_rst/dat:data.Column.set_not_empty}\pysiglinewithargsret{\bfcode{set\_not\_empty}}{}{}
Set Column to be not empty

\end{fulllineitems}

\index{set\_size() (data.Column method)}

\begin{fulllineitems}
\phantomsection\label{Code_rst/dat:data.Column.set_size}\pysiglinewithargsret{\bfcode{set\_size}}{\emph{size}}{}
Sets the size of the data for use when checking for errors.
For use with the `Identifier' data type

size -- length of identifier

\end{fulllineitems}

\index{set\_type() (data.Column method)}

\begin{fulllineitems}
\phantomsection\label{Code_rst/dat:data.Column.set_type}\pysiglinewithargsret{\bfcode{set\_type}}{\emph{type}}{}
Sets type of column for use with templates

\end{fulllineitems}

\index{updateCell() (data.Column method)}

\begin{fulllineitems}
\phantomsection\label{Code_rst/dat:data.Column.updateCell}\pysiglinewithargsret{\bfcode{updateCell}}{\emph{pos}, \emph{new\_value}}{}
Changes the value of a cell given

pos -- position of cell in column to change

new\_value -- value to set cell too

\end{fulllineitems}


\end{fulllineitems}

\index{Data (class in data)}

\begin{fulllineitems}
\phantomsection\label{Code_rst/dat:data.Data}\pysiglinewithargsret{\strong{class }\code{data.}\bfcode{Data}}{\emph{*args}}{}
Main store for CSV data, reading the data from the CSV file and then 
assigning out to relevant variables.
\begin{description}
\item[{Global variables:}] \leavevmode
threshold -- The percentage threshold which the column must have of a type before
it is declared that type.

enum\_threshold -- The integer threshold which if the count of occurence of a value is
less than the value is declared an error.

\item[{Methods:}] \leavevmode
read -- Reads the CSV file and outputs to raw\_data variable.

remove\_invalid -- Reads from raw\_data variable and assigns rows to 
valid\_rows or invalid\_rows according to their length.

create\_columns -- Creates column object according to valid\_rows, assigning
column header and column values.

clean -- Calls column cleaning methods to run `cleaning' on all columns.

analysis -- Calls column analysis methods to run `analysis' on all columns.

\end{description}

Variables:
\begin{quote}

analysers -- Dictionary conataining types as keys and their respective
analysers as values (i.e. analysers{[}'type'{]} == TypeAnalyser

types -- Tuple containing all valid types as ordered pairs of form 
(`Type', `Human readable type'). Used to map types on web site
to their correct progammatic name.

Filename -- String of path to file containing data

columns -- List of column objects.

invalid\_rows -- List of invalid rows (i.e., more or less columns than
number of headers). Copied from raw\_data

invalid\_rows\_indexes -- List of indexes corresponding to invalid rows.

formatted\_invalid\_rows -- List of invalid rows for report.

invalid\_rows\_pos -- List of the amount of invalid rows in the raw data prior
to each valid row (i.e. the nth element contains number of invalid rows
prior to the nth valid row)

errors -- list of errors in file; error{[}n{]}{[}0{]} is row of error, error{[}n{]}{[}1{]}
is column of error, and error{[}n{]}{[}2{]} is the value of in that location.

formatted\_errors -- List of errors in file, each error contains: row, column 
and value of the error.

raw\_data -- List of raw CSV data as rows. After remove\_invalid() has run
this only contains rows from the CSV file prior to the start of the data.

valid\_rows -- List of valid rows (i.e., same number of columns as headers).
\end{quote}
\index{analysis() (data.Data method)}

\begin{fulllineitems}
\phantomsection\label{Code_rst/dat:data.Data.analysis}\pysiglinewithargsret{\bfcode{analysis}}{}{}
Iterates through each column and analyses the columns values using the
columns type analyser.

\end{fulllineitems}

\index{change\_row() (data.Data method)}

\begin{fulllineitems}
\phantomsection\label{Code_rst/dat:data.Data.change_row}\pysiglinewithargsret{\bfcode{change\_row}}{\emph{row\_num}, \emph{new\_values}}{}
Edits a row of the data given:
row\_num - number of row being changed
new\_values - list of values row is to be changed to

\end{fulllineitems}

\index{clean() (data.Data method)}

\begin{fulllineitems}
\phantomsection\label{Code_rst/dat:data.Data.clean}\pysiglinewithargsret{\bfcode{clean}}{}{}
Calls cleaning methods on all columns.

\end{fulllineitems}

\index{clear\_errors() (data.Data method)}

\begin{fulllineitems}
\phantomsection\label{Code_rst/dat:data.Data.clear_errors}\pysiglinewithargsret{\bfcode{clear\_errors}}{}{}
Wipes recorded errors to allow find\_errors() to be rerun

\end{fulllineitems}

\index{create\_columns() (data.Data method)}

\begin{fulllineitems}
\phantomsection\label{Code_rst/dat:data.Data.create_columns}\pysiglinewithargsret{\bfcode{create\_columns}}{}{}
For each row in raw\_data variable, assigns the first value to the 
headers variable and creates a Column object with that header provided.
Then removes header row from valid\_rows. Then for each row in valid\_rows,
populates relevant column object with row data.

\end{fulllineitems}

\index{find\_errors() (data.Data method)}

\begin{fulllineitems}
\phantomsection\label{Code_rst/dat:data.Data.find_errors}\pysiglinewithargsret{\bfcode{find\_errors}}{}{}
Iterates through each column and finds any errors according to pre-determined
conditions.

\end{fulllineitems}

\index{gen\_file() (data.Data method)}

\begin{fulllineitems}
\phantomsection\label{Code_rst/dat:data.Data.gen_file}\pysiglinewithargsret{\bfcode{gen\_file}}{}{}
Generates a csv file based on the data for after
data has been corrected

\end{fulllineitems}

\index{getCellErrors() (data.Data method)}

\begin{fulllineitems}
\phantomsection\label{Code_rst/dat:data.Data.getCellErrors}\pysiglinewithargsret{\bfcode{getCellErrors}}{}{}
Returns list of all cells containing invalid data, contains
row number,. column number and its value.

\end{fulllineitems}

\index{getColumns() (data.Data method)}

\begin{fulllineitems}
\phantomsection\label{Code_rst/dat:data.Data.getColumns}\pysiglinewithargsret{\bfcode{getColumns}}{}{}
Returns a list of all columns

\end{fulllineitems}

\index{getRowErrors() (data.Data method)}

\begin{fulllineitems}
\phantomsection\label{Code_rst/dat:data.Data.getRowErrors}\pysiglinewithargsret{\bfcode{getRowErrors}}{}{}
Returns a list of all row errors

\end{fulllineitems}

\index{get\_column() (data.Data method)}

\begin{fulllineitems}
\phantomsection\label{Code_rst/dat:data.Data.get_column}\pysiglinewithargsret{\bfcode{get\_column}}{\emph{colNo}}{}
Returns a column of the data given a column number

\end{fulllineitems}

\index{get\_headers() (data.Data method)}

\begin{fulllineitems}
\phantomsection\label{Code_rst/dat:data.Data.get_headers}\pysiglinewithargsret{\bfcode{get\_headers}}{}{}
Returns the headers of data

\end{fulllineitems}

\index{get\_row() (data.Data method)}

\begin{fulllineitems}
\phantomsection\label{Code_rst/dat:data.Data.get_row}\pysiglinewithargsret{\bfcode{get\_row}}{\emph{row\_num}}{}
Returns the values of a row in list

\end{fulllineitems}

\index{pre\_analysis() (data.Data method)}

\begin{fulllineitems}
\phantomsection\label{Code_rst/dat:data.Data.pre_analysis}\pysiglinewithargsret{\bfcode{pre\_analysis}}{}{}
First defines their least and most common elements, then if 
template is supplied, sets the type of the column to match the template, if not if 
column is not empty defines its type, and if it's a special data type sets the columns
size to me no more than data\_size.

\end{fulllineitems}

\index{read() (data.Data method)}

\begin{fulllineitems}
\phantomsection\label{Code_rst/dat:data.Data.read}\pysiglinewithargsret{\bfcode{read}}{\emph{csv\_file}}{}
Opens and reads the CSV file, line by line, to raw\_data variable.
\begin{description}
\item[{Keyword arguments:}] \leavevmode
csv\_file -- The filename of the file to be opened.

\end{description}

\end{fulllineitems}

\index{remove\_invalid() (data.Data method)}

\begin{fulllineitems}
\phantomsection\label{Code_rst/dat:data.Data.remove_invalid}\pysiglinewithargsret{\bfcode{remove\_invalid}}{}{}
For each row in raw\_data variable, checks row length and appends to 
valid\_rows variable if same length as headers, else appends to 
invalid\_rows variable. invalid\_rows\_indexes holds the amount of rows that have been
skipped by the point the xth row has been accessed from valid\_rows.

\end{fulllineitems}

\index{set\_headers() (data.Data method)}

\begin{fulllineitems}
\phantomsection\label{Code_rst/dat:data.Data.set_headers}\pysiglinewithargsret{\bfcode{set\_headers}}{\emph{header\_map}}{}
Sets headers of columns taking a dictionary 
mapping column numbers to headers

\end{fulllineitems}


\end{fulllineitems}



\section{Report}
\label{Code_rst/rep:report}\label{Code_rst/rep::doc}\label{Code_rst/rep:module-report}\index{report (module)}
Generate reports based on data provided via a Data object.

Classes:
Report -- Contains methods to generate and output appropriate HTML for the report.
\index{Report (class in report)}

\begin{fulllineitems}
\phantomsection\label{Code_rst/rep:report.Report}\pysiglinewithargsret{\strong{class }\code{report.}\bfcode{Report}}{\emph{data}, \emph{file}}{}
The main report object.
\begin{description}
\item[{Methods:}] \leavevmode
\_\_init\_\_ -- Initialise the object and create required local variables.

empty\_columns -- Return empty columns in the data object.

html\_report -- Create HTML report and output to file.

list\_creator -- Helper method to generate a HTML list from provided input.

row\_creator -- Helper method to generate HTML rows from provided input.

numerical\_analysis -- Return numerical based statistics on input.

string\_analysis -- Return string based statistics on input.

enum\_analysis -- Return enumeration based statistics on input.

bool\_analysis -- Return boolean based statistics on input.

email\_analysis -- Return email based statistics on input.

date\_analysis -- Return date based statistics on input.

\item[{Variables:}] \leavevmode
data -- Reference to Data object.

file -- Reference to CSV file.

chart\_data -- The javascript strings that need to be added to the template for the charts to display.

\end{description}
\index{boolean\_analysis() (report.Report method)}

\begin{fulllineitems}
\phantomsection\label{Code_rst/rep:report.Report.boolean_analysis}\pysiglinewithargsret{\bfcode{boolean\_analysis}}{}{}
Return HTML string of boolean analysis on columns of type boolean
in the data objectby accessing the various class variables of the
columns.

\end{fulllineitems}

\index{char\_analysis() (report.Report method)}

\begin{fulllineitems}
\phantomsection\label{Code_rst/rep:report.Report.char_analysis}\pysiglinewithargsret{\bfcode{char\_analysis}}{}{}
Return HTML string of char analysis on columns of type char 
in the data object by accessing the various class variables of the
columns.

\end{fulllineitems}

\index{currency\_analysis() (report.Report method)}

\begin{fulllineitems}
\phantomsection\label{Code_rst/rep:report.Report.currency_analysis}\pysiglinewithargsret{\bfcode{currency\_analysis}}{}{}~\begin{description}
\item[{Return HTML string of numerical analysis on columns of type Currency}] \leavevmode
in the data object by accessing the various class variables of the

\end{description}

columns.

\end{fulllineitems}

\index{date\_analysis() (report.Report method)}

\begin{fulllineitems}
\phantomsection\label{Code_rst/rep:report.Report.date_analysis}\pysiglinewithargsret{\bfcode{date\_analysis}}{}{}
Return HTML string of date analysis on columns of type date 
in the data object by accessing the various class variables of the
columns.

\end{fulllineitems}

\index{day\_analysis() (report.Report method)}

\begin{fulllineitems}
\phantomsection\label{Code_rst/rep:report.Report.day_analysis}\pysiglinewithargsret{\bfcode{day\_analysis}}{}{}
Return HTML string of day analysis on columns of type day 
in the data object by accessing the various class variables of the
columns.

\end{fulllineitems}

\index{email\_analysis() (report.Report method)}

\begin{fulllineitems}
\phantomsection\label{Code_rst/rep:report.Report.email_analysis}\pysiglinewithargsret{\bfcode{email\_analysis}}{}{}
Return HTML string of email analysis on columns of type email
in the data objectby accessing the various class variables of the
columns.

\end{fulllineitems}

\index{empty\_columns() (report.Report method)}

\begin{fulllineitems}
\phantomsection\label{Code_rst/rep:report.Report.empty_columns}\pysiglinewithargsret{\bfcode{empty\_columns}}{}{}
Return a list of empty rows in the data object.

\end{fulllineitems}

\index{enum\_analysis() (report.Report method)}

\begin{fulllineitems}
\phantomsection\label{Code_rst/rep:report.Report.enum_analysis}\pysiglinewithargsret{\bfcode{enum\_analysis}}{}{}
Return HTML string of enumeration analysis on columns of type Enum 
in the data object by accessing the various class variables of the
columns.

\end{fulllineitems}

\index{gen\_html() (report.Report method)}

\begin{fulllineitems}
\phantomsection\label{Code_rst/rep:report.Report.gen_html}\pysiglinewithargsret{\bfcode{gen\_html}}{\emph{html}}{}
Generates html report for the file

\end{fulllineitems}

\index{html\_report() (report.Report method)}

\begin{fulllineitems}
\phantomsection\label{Code_rst/rep:report.Report.html_report}\pysiglinewithargsret{\bfcode{html\_report}}{}{}
Write a HTML file based on analysis of CSV file by calling the various
type analyses.
\begin{quote}

Returns string of html report
\end{quote}

\end{fulllineitems}

\index{hyper\_analysis() (report.Report method)}

\begin{fulllineitems}
\phantomsection\label{Code_rst/rep:report.Report.hyper_analysis}\pysiglinewithargsret{\bfcode{hyper\_analysis}}{}{}
Return HTML string of hyperlink analysis on columns of type hyper 
in the data object by accessing the various class variables of the
columns.

\end{fulllineitems}

\index{identifier\_analysis() (report.Report method)}

\begin{fulllineitems}
\phantomsection\label{Code_rst/rep:report.Report.identifier_analysis}\pysiglinewithargsret{\bfcode{identifier\_analysis}}{}{}
Return HTML string of identifier analysis on columns of type identifier
in the data objectby accessing the various class variables of the
columns by accessing the various class variables of the
columns.

\end{fulllineitems}

\index{initial\_show\_items() (report.Report static method)}

\begin{fulllineitems}
\phantomsection\label{Code_rst/rep:report.Report.initial_show_items}\pysiglinewithargsret{\strong{static }\bfcode{initial\_show\_items}}{}{}
Return the number of items to show initially, where clicking `show more' will expand.

\end{fulllineitems}

\index{list\_creator() (report.Report static method)}

\begin{fulllineitems}
\phantomsection\label{Code_rst/rep:report.Report.list_creator}\pysiglinewithargsret{\strong{static }\bfcode{list\_creator}}{\emph{list\_items}}{}
Return provided list as an unordered HTML list.

Keyword arguments:
list\_items -- List of items to be turned into HTML.

\end{fulllineitems}

\index{numerical\_analysis() (report.Report method)}

\begin{fulllineitems}
\phantomsection\label{Code_rst/rep:report.Report.numerical_analysis}\pysiglinewithargsret{\bfcode{numerical\_analysis}}{}{}
Return HTML string of numerical analysis on columns of type Float or 
Integer in the data object by accessing the various class variables of the
columns.

\end{fulllineitems}

\index{row\_creator() (report.Report static method)}

\begin{fulllineitems}
\phantomsection\label{Code_rst/rep:report.Report.row_creator}\pysiglinewithargsret{\strong{static }\bfcode{row\_creator}}{\emph{row\_items}, \emph{rowNumber=0}, \emph{type='none'}}{}
Return provided list as HTML rows.

Arguments:
row\_items -- List of items to be turned into HTML.

\end{fulllineitems}

\index{string\_analysis() (report.Report method)}

\begin{fulllineitems}
\phantomsection\label{Code_rst/rep:report.Report.string_analysis}\pysiglinewithargsret{\bfcode{string\_analysis}}{}{}
Return HTML string of string analysis on columns of type string 
in the data object by accessing the various class variables of the
columns.

\end{fulllineitems}

\index{time\_analysis() (report.Report method)}

\begin{fulllineitems}
\phantomsection\label{Code_rst/rep:report.Report.time_analysis}\pysiglinewithargsret{\bfcode{time\_analysis}}{}{}
Return HTML string of time analysis on columns of type time 
in the data object by accessing the various class variables of the
columns.

\end{fulllineitems}


\end{fulllineitems}



\section{Template}
\label{Code_rst/templ:module-template}\label{Code_rst/templ:template}\label{Code_rst/templ::doc}\index{template (module)}
Provide a base HTML template variable for population with appropriate
statistics in the report.py module.


\section{Analyser}
\label{Code_rst/analy::doc}\label{Code_rst/analy:analyser}

\section{Template Reader}
\label{Code_rst/temp_rd:template-reader}\label{Code_rst/temp_rd::doc}\label{Code_rst/temp_rd:module-template_reader}\index{template\_reader (module)}
Class for reading templates to pass on information about how
to process the data for the data class
\index{Template (class in template\_reader)}

\begin{fulllineitems}
\phantomsection\label{Code_rst/temp_rd:template_reader.Template}\pysiglinewithargsret{\strong{class }\code{template\_reader.}\bfcode{Template}}{\emph{filename}}{}~\begin{description}
\item[{Object storing user input that describes data given. Able to specify:}] \leavevmode
Columns - state column number and data type.

Delimiter - state delimiter character (for comma and space use the word not `,' or ` `).

Header row - row of header (0 for no header).

Start row - row that data starts on

Threshold value - minimum proportion of column that has the correct data type

\end{description}

Columns and rows start at 1 not 0
\index{read() (template\_reader.Template method)}

\begin{fulllineitems}
\phantomsection\label{Code_rst/temp_rd:template_reader.Template.read}\pysiglinewithargsret{\bfcode{read}}{\emph{filename}}{}
Reads template file, assumes correct formatting, if user editing
is permitted will need to be improved with more checks

See documentation

\end{fulllineitems}


\end{fulllineitems}



\chapter{Indices and tables}
\label{index:indices-and-tables}\begin{itemize}
\item {} 
\DUspan{xref,std,std-ref}{genindex}

\item {} 
\DUspan{xref,std,std-ref}{modindex}

\item {} 
\DUspan{xref,std,std-ref}{search}

\end{itemize}


\renewcommand{\indexname}{Python Module Index}
\begin{theindex}
\def\bigletter#1{{\Large\sffamily#1}\nopagebreak\vspace{1mm}}
\bigletter{a}
\item {\texttt{application}}, \pageref{Code_rst/app:module-application}
\indexspace
\bigletter{d}
\item {\texttt{data}}, \pageref{Code_rst/dat:module-data}
\indexspace
\bigletter{r}
\item {\texttt{report}}, \pageref{Code_rst/rep:module-report}
\indexspace
\bigletter{t}
\item {\texttt{template}}, \pageref{Code_rst/templ:module-template}
\item {\texttt{template\_reader}}, \pageref{Code_rst/temp_rd:module-template_reader}
\end{theindex}

\renewcommand{\indexname}{Index}
\printindex
\end{document}
