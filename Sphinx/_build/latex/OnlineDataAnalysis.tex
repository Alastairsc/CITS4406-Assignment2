% Generated by Sphinx.
\def\sphinxdocclass{report}
\documentclass[letterpaper,10pt,english]{sphinxmanual}
\usepackage[utf8]{inputenc}
\DeclareUnicodeCharacter{00A0}{\nobreakspace}
\usepackage{cmap}
\usepackage[T1]{fontenc}
\usepackage{babel}
\usepackage{times}
\usepackage[Bjarne]{fncychap}
\usepackage{longtable}
\usepackage{sphinx}
\usepackage{multirow}

\addto\captionsenglish{\renewcommand{\figurename}{Fig. }}
\addto\captionsenglish{\renewcommand{\tablename}{Table }}
\floatname{literal-block}{Listing }



\title{Online Data Analysis Documentation}
\date{October 15, 2015}
\release{1.0.0}
\author{Alastair Chin}
\newcommand{\sphinxlogo}{}
\renewcommand{\releasename}{Release}
\makeindex

\makeatletter
\def\PYG@reset{\let\PYG@it=\relax \let\PYG@bf=\relax%
    \let\PYG@ul=\relax \let\PYG@tc=\relax%
    \let\PYG@bc=\relax \let\PYG@ff=\relax}
\def\PYG@tok#1{\csname PYG@tok@#1\endcsname}
\def\PYG@toks#1+{\ifx\relax#1\empty\else%
    \PYG@tok{#1}\expandafter\PYG@toks\fi}
\def\PYG@do#1{\PYG@bc{\PYG@tc{\PYG@ul{%
    \PYG@it{\PYG@bf{\PYG@ff{#1}}}}}}}
\def\PYG#1#2{\PYG@reset\PYG@toks#1+\relax+\PYG@do{#2}}

\expandafter\def\csname PYG@tok@sx\endcsname{\def\PYG@tc##1{\textcolor[rgb]{0.78,0.36,0.04}{##1}}}
\expandafter\def\csname PYG@tok@nl\endcsname{\let\PYG@bf=\textbf\def\PYG@tc##1{\textcolor[rgb]{0.00,0.13,0.44}{##1}}}
\expandafter\def\csname PYG@tok@sb\endcsname{\def\PYG@tc##1{\textcolor[rgb]{0.25,0.44,0.63}{##1}}}
\expandafter\def\csname PYG@tok@go\endcsname{\def\PYG@tc##1{\textcolor[rgb]{0.20,0.20,0.20}{##1}}}
\expandafter\def\csname PYG@tok@s2\endcsname{\def\PYG@tc##1{\textcolor[rgb]{0.25,0.44,0.63}{##1}}}
\expandafter\def\csname PYG@tok@gt\endcsname{\def\PYG@tc##1{\textcolor[rgb]{0.00,0.27,0.87}{##1}}}
\expandafter\def\csname PYG@tok@sc\endcsname{\def\PYG@tc##1{\textcolor[rgb]{0.25,0.44,0.63}{##1}}}
\expandafter\def\csname PYG@tok@vg\endcsname{\def\PYG@tc##1{\textcolor[rgb]{0.73,0.38,0.84}{##1}}}
\expandafter\def\csname PYG@tok@m\endcsname{\def\PYG@tc##1{\textcolor[rgb]{0.13,0.50,0.31}{##1}}}
\expandafter\def\csname PYG@tok@no\endcsname{\def\PYG@tc##1{\textcolor[rgb]{0.38,0.68,0.84}{##1}}}
\expandafter\def\csname PYG@tok@gr\endcsname{\def\PYG@tc##1{\textcolor[rgb]{1.00,0.00,0.00}{##1}}}
\expandafter\def\csname PYG@tok@cs\endcsname{\def\PYG@tc##1{\textcolor[rgb]{0.25,0.50,0.56}{##1}}\def\PYG@bc##1{\setlength{\fboxsep}{0pt}\colorbox[rgb]{1.00,0.94,0.94}{\strut ##1}}}
\expandafter\def\csname PYG@tok@cp\endcsname{\def\PYG@tc##1{\textcolor[rgb]{0.00,0.44,0.13}{##1}}}
\expandafter\def\csname PYG@tok@vc\endcsname{\def\PYG@tc##1{\textcolor[rgb]{0.73,0.38,0.84}{##1}}}
\expandafter\def\csname PYG@tok@mo\endcsname{\def\PYG@tc##1{\textcolor[rgb]{0.13,0.50,0.31}{##1}}}
\expandafter\def\csname PYG@tok@gi\endcsname{\def\PYG@tc##1{\textcolor[rgb]{0.00,0.63,0.00}{##1}}}
\expandafter\def\csname PYG@tok@gu\endcsname{\let\PYG@bf=\textbf\def\PYG@tc##1{\textcolor[rgb]{0.50,0.00,0.50}{##1}}}
\expandafter\def\csname PYG@tok@kd\endcsname{\let\PYG@bf=\textbf\def\PYG@tc##1{\textcolor[rgb]{0.00,0.44,0.13}{##1}}}
\expandafter\def\csname PYG@tok@nt\endcsname{\let\PYG@bf=\textbf\def\PYG@tc##1{\textcolor[rgb]{0.02,0.16,0.45}{##1}}}
\expandafter\def\csname PYG@tok@il\endcsname{\def\PYG@tc##1{\textcolor[rgb]{0.13,0.50,0.31}{##1}}}
\expandafter\def\csname PYG@tok@nb\endcsname{\def\PYG@tc##1{\textcolor[rgb]{0.00,0.44,0.13}{##1}}}
\expandafter\def\csname PYG@tok@nd\endcsname{\let\PYG@bf=\textbf\def\PYG@tc##1{\textcolor[rgb]{0.33,0.33,0.33}{##1}}}
\expandafter\def\csname PYG@tok@w\endcsname{\def\PYG@tc##1{\textcolor[rgb]{0.73,0.73,0.73}{##1}}}
\expandafter\def\csname PYG@tok@mh\endcsname{\def\PYG@tc##1{\textcolor[rgb]{0.13,0.50,0.31}{##1}}}
\expandafter\def\csname PYG@tok@s1\endcsname{\def\PYG@tc##1{\textcolor[rgb]{0.25,0.44,0.63}{##1}}}
\expandafter\def\csname PYG@tok@gp\endcsname{\let\PYG@bf=\textbf\def\PYG@tc##1{\textcolor[rgb]{0.78,0.36,0.04}{##1}}}
\expandafter\def\csname PYG@tok@si\endcsname{\let\PYG@it=\textit\def\PYG@tc##1{\textcolor[rgb]{0.44,0.63,0.82}{##1}}}
\expandafter\def\csname PYG@tok@k\endcsname{\let\PYG@bf=\textbf\def\PYG@tc##1{\textcolor[rgb]{0.00,0.44,0.13}{##1}}}
\expandafter\def\csname PYG@tok@err\endcsname{\def\PYG@bc##1{\setlength{\fboxsep}{0pt}\fcolorbox[rgb]{1.00,0.00,0.00}{1,1,1}{\strut ##1}}}
\expandafter\def\csname PYG@tok@ni\endcsname{\let\PYG@bf=\textbf\def\PYG@tc##1{\textcolor[rgb]{0.84,0.33,0.22}{##1}}}
\expandafter\def\csname PYG@tok@mi\endcsname{\def\PYG@tc##1{\textcolor[rgb]{0.13,0.50,0.31}{##1}}}
\expandafter\def\csname PYG@tok@kt\endcsname{\def\PYG@tc##1{\textcolor[rgb]{0.56,0.13,0.00}{##1}}}
\expandafter\def\csname PYG@tok@gh\endcsname{\let\PYG@bf=\textbf\def\PYG@tc##1{\textcolor[rgb]{0.00,0.00,0.50}{##1}}}
\expandafter\def\csname PYG@tok@c1\endcsname{\let\PYG@it=\textit\def\PYG@tc##1{\textcolor[rgb]{0.25,0.50,0.56}{##1}}}
\expandafter\def\csname PYG@tok@ge\endcsname{\let\PYG@it=\textit}
\expandafter\def\csname PYG@tok@kc\endcsname{\let\PYG@bf=\textbf\def\PYG@tc##1{\textcolor[rgb]{0.00,0.44,0.13}{##1}}}
\expandafter\def\csname PYG@tok@vi\endcsname{\def\PYG@tc##1{\textcolor[rgb]{0.73,0.38,0.84}{##1}}}
\expandafter\def\csname PYG@tok@sd\endcsname{\let\PYG@it=\textit\def\PYG@tc##1{\textcolor[rgb]{0.25,0.44,0.63}{##1}}}
\expandafter\def\csname PYG@tok@na\endcsname{\def\PYG@tc##1{\textcolor[rgb]{0.25,0.44,0.63}{##1}}}
\expandafter\def\csname PYG@tok@mb\endcsname{\def\PYG@tc##1{\textcolor[rgb]{0.13,0.50,0.31}{##1}}}
\expandafter\def\csname PYG@tok@nc\endcsname{\let\PYG@bf=\textbf\def\PYG@tc##1{\textcolor[rgb]{0.05,0.52,0.71}{##1}}}
\expandafter\def\csname PYG@tok@s\endcsname{\def\PYG@tc##1{\textcolor[rgb]{0.25,0.44,0.63}{##1}}}
\expandafter\def\csname PYG@tok@ss\endcsname{\def\PYG@tc##1{\textcolor[rgb]{0.32,0.47,0.09}{##1}}}
\expandafter\def\csname PYG@tok@nv\endcsname{\def\PYG@tc##1{\textcolor[rgb]{0.73,0.38,0.84}{##1}}}
\expandafter\def\csname PYG@tok@ow\endcsname{\let\PYG@bf=\textbf\def\PYG@tc##1{\textcolor[rgb]{0.00,0.44,0.13}{##1}}}
\expandafter\def\csname PYG@tok@o\endcsname{\def\PYG@tc##1{\textcolor[rgb]{0.40,0.40,0.40}{##1}}}
\expandafter\def\csname PYG@tok@ne\endcsname{\def\PYG@tc##1{\textcolor[rgb]{0.00,0.44,0.13}{##1}}}
\expandafter\def\csname PYG@tok@nf\endcsname{\def\PYG@tc##1{\textcolor[rgb]{0.02,0.16,0.49}{##1}}}
\expandafter\def\csname PYG@tok@kp\endcsname{\def\PYG@tc##1{\textcolor[rgb]{0.00,0.44,0.13}{##1}}}
\expandafter\def\csname PYG@tok@nn\endcsname{\let\PYG@bf=\textbf\def\PYG@tc##1{\textcolor[rgb]{0.05,0.52,0.71}{##1}}}
\expandafter\def\csname PYG@tok@sr\endcsname{\def\PYG@tc##1{\textcolor[rgb]{0.14,0.33,0.53}{##1}}}
\expandafter\def\csname PYG@tok@sh\endcsname{\def\PYG@tc##1{\textcolor[rgb]{0.25,0.44,0.63}{##1}}}
\expandafter\def\csname PYG@tok@kn\endcsname{\let\PYG@bf=\textbf\def\PYG@tc##1{\textcolor[rgb]{0.00,0.44,0.13}{##1}}}
\expandafter\def\csname PYG@tok@gs\endcsname{\let\PYG@bf=\textbf}
\expandafter\def\csname PYG@tok@c\endcsname{\let\PYG@it=\textit\def\PYG@tc##1{\textcolor[rgb]{0.25,0.50,0.56}{##1}}}
\expandafter\def\csname PYG@tok@cm\endcsname{\let\PYG@it=\textit\def\PYG@tc##1{\textcolor[rgb]{0.25,0.50,0.56}{##1}}}
\expandafter\def\csname PYG@tok@bp\endcsname{\def\PYG@tc##1{\textcolor[rgb]{0.00,0.44,0.13}{##1}}}
\expandafter\def\csname PYG@tok@se\endcsname{\let\PYG@bf=\textbf\def\PYG@tc##1{\textcolor[rgb]{0.25,0.44,0.63}{##1}}}
\expandafter\def\csname PYG@tok@kr\endcsname{\let\PYG@bf=\textbf\def\PYG@tc##1{\textcolor[rgb]{0.00,0.44,0.13}{##1}}}
\expandafter\def\csname PYG@tok@mf\endcsname{\def\PYG@tc##1{\textcolor[rgb]{0.13,0.50,0.31}{##1}}}
\expandafter\def\csname PYG@tok@gd\endcsname{\def\PYG@tc##1{\textcolor[rgb]{0.63,0.00,0.00}{##1}}}

\def\PYGZbs{\char`\\}
\def\PYGZus{\char`\_}
\def\PYGZob{\char`\{}
\def\PYGZcb{\char`\}}
\def\PYGZca{\char`\^}
\def\PYGZam{\char`\&}
\def\PYGZlt{\char`\<}
\def\PYGZgt{\char`\>}
\def\PYGZsh{\char`\#}
\def\PYGZpc{\char`\%}
\def\PYGZdl{\char`\$}
\def\PYGZhy{\char`\-}
\def\PYGZsq{\char`\'}
\def\PYGZdq{\char`\"}
\def\PYGZti{\char`\~}
% for compatibility with earlier versions
\def\PYGZat{@}
\def\PYGZlb{[}
\def\PYGZrb{]}
\makeatother

\renewcommand\PYGZsq{\textquotesingle}

\begin{document}

\maketitle
\tableofcontents
\phantomsection\label{index::doc}


\includegraphics{UWA-logo.png}

This contains the documentation for the online data analysis project


\chapter{Requirements}
\label{index:welcome-to-online-data-analysis-s-documentation}\label{index:requirements}
This requires Python 3, scipy, numpy and pandas which can be installed
by using the Anaconda package which contains all necessary files.

The program currently consists of 6 files

Contents:


\section{Application}
\label{Code_rst/app:module-application}\label{Code_rst/app::doc}\label{Code_rst/app:application}\index{application (module)}
Main execution body for program.
\index{get\_file\_dir() (in module application)}

\begin{fulllineitems}
\phantomsection\label{Code_rst/app:application.get_file_dir}\pysiglinewithargsret{\code{application.}\bfcode{get\_file\_dir}}{\emph{location}}{}
Returns the directory of the file with the file name
\begin{description}
\item[{Keyword arguments:}] \leavevmode
location -- A file path.

\end{description}

\end{fulllineitems}

\index{main() (in module application)}

\begin{fulllineitems}
\phantomsection\label{Code_rst/app:application.main}\pysiglinewithargsret{\code{application.}\bfcode{main}}{\emph{*args}}{}
Create Data and Report objects, providing necessary information for them 
to run analysis and create desired outputs (i.e. HTML report).

\end{fulllineitems}



\section{Data}
\label{Code_rst/dat:data}\label{Code_rst/dat::doc}

\section{Report}
\label{Code_rst/rep:report}\label{Code_rst/rep::doc}\label{Code_rst/rep:module-report}\index{report (module)}
Generate reports based on data provided via a Data object.
\begin{description}
\item[{Classes:}] \leavevmode
Report -- Contains methods to generate and output appropriate HTML for the report.

\end{description}
\index{Report (class in report)}

\begin{fulllineitems}
\phantomsection\label{Code_rst/rep:report.Report}\pysiglinewithargsret{\strong{class }\code{report.}\bfcode{Report}}{\emph{data}, \emph{file}}{}
The main report object.
\begin{description}
\item[{Methods:}] \leavevmode
\_\_init\_\_ -- Initialise the object and create required local variables.

empty\_columns -- Return empty columns in the data object.

html\_report -- Create HTML report and output to file.

list\_creator -- Helper method to generate a HTML list from provided input.

row\_creator -- Helper method to generate HTML rows from provided input.

numerical\_analysis -- Return numerical based statistics on input.

string\_analysis -- Return string based statistics on input.

enum\_analysis -- Return enumeration based statistics on input.

bool\_analysis -- Return boolean based statistics on input.

email\_analysis -- Return email based statistics on input.

date\_analysis -- Return date based statistics on input.

time\_anaysis -- Return time based statistics on input.

day\_analysis -- Return day based statistics on input.

hyper\_analysis -- Return hyperlink based analyis on input.

list\_creator -- Provided a list, returns an unordered html list of values in the list.

\item[{Static Methods:}] \leavevmode
list\_creator -- Provided a list, returns an unordered html list of values in the list.

row\_creator -- Provided a list, returns a HTML row of values in the list.

initial\_show\_items -- Returns the number of items to show initially for each type in the 
report before hiding them under a `show more' button.

\item[{Variables:}] \leavevmode
GRAPH\_LIMIT -- How many rows before the graphs will stop displaying every value, but just show a summary

data -- Reference to Data object, the output from analysis of file\_name.

file\_name -- Reference to a CSV file containing the data being worked on.

chart\_data -- A String of data that is formatted correctly to be input to a graph API.

\end{description}
\index{boolean\_analysis() (report.Report method)}

\begin{fulllineitems}
\phantomsection\label{Code_rst/rep:report.Report.boolean_analysis}\pysiglinewithargsret{\bfcode{boolean\_analysis}}{}{}
Return HTML string of boolean analysis on columns of type boolean
in the data objectby accessing the various class variables of the
columns.

\end{fulllineitems}

\index{char\_analysis() (report.Report method)}

\begin{fulllineitems}
\phantomsection\label{Code_rst/rep:report.Report.char_analysis}\pysiglinewithargsret{\bfcode{char\_analysis}}{}{}
Return HTML string of char analysis on columns of type char 
in the data object by accessing the various class variables of the
columns.

\end{fulllineitems}

\index{currency\_analysis() (report.Report method)}

\begin{fulllineitems}
\phantomsection\label{Code_rst/rep:report.Report.currency_analysis}\pysiglinewithargsret{\bfcode{currency\_analysis}}{}{}
Return HTML string of numerical analysis on columns of type Currency
in the data object by accessing the various class variables of the
columns.

\end{fulllineitems}

\index{date\_analysis() (report.Report method)}

\begin{fulllineitems}
\phantomsection\label{Code_rst/rep:report.Report.date_analysis}\pysiglinewithargsret{\bfcode{date\_analysis}}{}{}
Return HTML string of date analysis on columns of type date 
in the data object by accessing the various class variables of the
columns.

\end{fulllineitems}

\index{day\_analysis() (report.Report method)}

\begin{fulllineitems}
\phantomsection\label{Code_rst/rep:report.Report.day_analysis}\pysiglinewithargsret{\bfcode{day\_analysis}}{}{}
Return HTML string of day analysis on columns of type day 
in the data object by accessing the various class variables of the
columns.

\end{fulllineitems}

\index{email\_analysis() (report.Report method)}

\begin{fulllineitems}
\phantomsection\label{Code_rst/rep:report.Report.email_analysis}\pysiglinewithargsret{\bfcode{email\_analysis}}{}{}
Return HTML string of email analysis on columns of type email
in the data objectby accessing the various class variables of the
columns.

\end{fulllineitems}

\index{empty\_columns() (report.Report method)}

\begin{fulllineitems}
\phantomsection\label{Code_rst/rep:report.Report.empty_columns}\pysiglinewithargsret{\bfcode{empty\_columns}}{}{}
Return a list of empty columns in the data object.

\end{fulllineitems}

\index{enum\_analysis() (report.Report method)}

\begin{fulllineitems}
\phantomsection\label{Code_rst/rep:report.Report.enum_analysis}\pysiglinewithargsret{\bfcode{enum\_analysis}}{}{}
Return HTML string of enumeration analysis on columns of type Enum 
in the data object by accessing the various class variables of the
columns.

\end{fulllineitems}

\index{gen\_html() (report.Report method)}

\begin{fulllineitems}
\phantomsection\label{Code_rst/rep:report.Report.gen_html}\pysiglinewithargsret{\bfcode{gen\_html}}{\emph{html}}{}
Generates html report for the file

\end{fulllineitems}

\index{html\_report() (report.Report method)}

\begin{fulllineitems}
\phantomsection\label{Code_rst/rep:report.Report.html_report}\pysiglinewithargsret{\bfcode{html\_report}}{}{}
Write a HTML file based on analysis of CSV file by calling the various
type analyses. Returns a string of html.

\end{fulllineitems}

\index{hyper\_analysis() (report.Report method)}

\begin{fulllineitems}
\phantomsection\label{Code_rst/rep:report.Report.hyper_analysis}\pysiglinewithargsret{\bfcode{hyper\_analysis}}{}{}
Return HTML string of hyperlink analysis on columns of type hyper 
in the data object by accessing the various class variables of the
columns.

\end{fulllineitems}

\index{identifier\_analysis() (report.Report method)}

\begin{fulllineitems}
\phantomsection\label{Code_rst/rep:report.Report.identifier_analysis}\pysiglinewithargsret{\bfcode{identifier\_analysis}}{}{}
Return HTML string of identifier analysis on columns of type identifier
in the data objectby accessing the various class variables of the
columns by accessing the various class variables of the
columns.

\end{fulllineitems}

\index{initial\_show\_items() (report.Report static method)}

\begin{fulllineitems}
\phantomsection\label{Code_rst/rep:report.Report.initial_show_items}\pysiglinewithargsret{\strong{static }\bfcode{initial\_show\_items}}{}{}
Return the number of items to show initially, where clicking `show more' will expand.

\end{fulllineitems}

\index{list\_creator() (report.Report static method)}

\begin{fulllineitems}
\phantomsection\label{Code_rst/rep:report.Report.list_creator}\pysiglinewithargsret{\strong{static }\bfcode{list\_creator}}{\emph{list\_items}}{}
Return provided list as an unordered HTML list.

Keyword arguments:
list\_items -- List of items to be turned into HTML.

\end{fulllineitems}

\index{numerical\_analysis() (report.Report method)}

\begin{fulllineitems}
\phantomsection\label{Code_rst/rep:report.Report.numerical_analysis}\pysiglinewithargsret{\bfcode{numerical\_analysis}}{}{}
Return HTML string of numerical analysis on columns of type Float or 
Integer in the data object by accessing the various class variables of the
columns.

\end{fulllineitems}

\index{row\_creator() (report.Report static method)}

\begin{fulllineitems}
\phantomsection\label{Code_rst/rep:report.Report.row_creator}\pysiglinewithargsret{\strong{static }\bfcode{row\_creator}}{\emph{row\_items}, \emph{rowNumber=0}, \emph{type='none'}}{}
Return provided list as HTML rows.

Arguments:
row\_items -- List of items to be turned into HTML.

\end{fulllineitems}

\index{string\_analysis() (report.Report method)}

\begin{fulllineitems}
\phantomsection\label{Code_rst/rep:report.Report.string_analysis}\pysiglinewithargsret{\bfcode{string\_analysis}}{}{}
Return HTML string of string analysis on columns of type string 
in the data object by accessing the various class variables of the
columns.

\end{fulllineitems}

\index{time\_analysis() (report.Report method)}

\begin{fulllineitems}
\phantomsection\label{Code_rst/rep:report.Report.time_analysis}\pysiglinewithargsret{\bfcode{time\_analysis}}{}{}
Return HTML string of time analysis on columns of type time 
in the data object by accessing the various class variables of the
columns.

\end{fulllineitems}


\end{fulllineitems}



\section{Template}
\label{Code_rst/templ:module-template}\label{Code_rst/templ:template}\label{Code_rst/templ::doc}\index{template (module)}
Provide a base HTML template variable for population with appropriate
statistics in the report.py module.


\section{Analyser}
\label{Code_rst/analy:analyser}\label{Code_rst/analy::doc}

\section{Template Reader}
\label{Code_rst/temp_rd:module-template_reader}\label{Code_rst/temp_rd::doc}\label{Code_rst/temp_rd:template-reader}\index{template\_reader (module)}
Class for reading templates to pass on information about how
to process the data for the data class
\index{Template (class in template\_reader)}

\begin{fulllineitems}
\phantomsection\label{Code_rst/temp_rd:template_reader.Template}\pysiglinewithargsret{\strong{class }\code{template\_reader.}\bfcode{Template}}{\emph{filename}}{}~\begin{description}
\item[{Object storing user input that describes data given. Able to specify:}] \leavevmode
Columns - state column number and data type.

Delimiter - state delimiter character (for comma and space use the word not `,' or ` `).

Header row - row of header (0 for no header).

Start row - row that data starts on

Threshold value - minimum proportion of column that has the correct data type

\end{description}

Columns and rows start at 1 not 0
\index{read() (template\_reader.Template method)}

\begin{fulllineitems}
\phantomsection\label{Code_rst/temp_rd:template_reader.Template.read}\pysiglinewithargsret{\bfcode{read}}{\emph{filename}}{}
Reads template file, assumes correct formatting, if user editing
is permitted will need to be improved with more checks

See documentation

\end{fulllineitems}


\end{fulllineitems}



\chapter{Indices and tables}
\label{index:indices-and-tables}\begin{itemize}
\item {} 
\DUspan{xref,std,std-ref}{genindex}

\item {} 
\DUspan{xref,std,std-ref}{modindex}

\item {} 
\DUspan{xref,std,std-ref}{search}

\end{itemize}


\renewcommand{\indexname}{Python Module Index}
\begin{theindex}
\def\bigletter#1{{\Large\sffamily#1}\nopagebreak\vspace{1mm}}
\bigletter{a}
\item {\texttt{application}}, \pageref{Code_rst/app:module-application}
\indexspace
\bigletter{r}
\item {\texttt{report}}, \pageref{Code_rst/rep:module-report}
\indexspace
\bigletter{t}
\item {\texttt{template}}, \pageref{Code_rst/templ:module-template}
\item {\texttt{template\_reader}}, \pageref{Code_rst/temp_rd:module-template_reader}
\end{theindex}

\renewcommand{\indexname}{Index}
\printindex
\end{document}
